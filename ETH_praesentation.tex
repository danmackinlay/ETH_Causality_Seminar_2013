% !TEX TS-program = pdflatex
% !TEX encoding = Isolatin

% Copyright Marius Hofert, Markus Kohm
% changed by D. Hennig, IT-Service ETH

%% Version 1.4 09.05.2011
% !TEX TS-program = pdflatex
% !TEX encoding = Isolatin
% !TeX root = ..\ETH_praesentation.tex


%% Version 1.2 21.04.2011
% ==== class ===================================================================
\documentclass[
	paper=128mm:96mm,  % like beamer
	fontsize=11pt,     % like beamer
	pagesize,          % write page size to dvi or pdf
	parskip=half-,     % paragraphs are separated by half a line, no special marking of line
                       % endings
	numbers=noendperiod,% removes points for special parts (e.g. appendix)
	captions=nooneline % do not distinguish between one or more lines in captions
    ]{scrartcl}
\linespread{1.12}% enlarge line space
% ==== packages ================================================================

% font stuff
\usepackage[T1]{fontenc}% for correct hyphenation and T1 encoding
\AfterFile{t1lmss.fd}{  % see http://www.komascript.de/node/823
  \DeclareFontShape{T1}{lmss}{b}{n}
  {<->ssub*lmss/bx/n}{}
}
\usepackage{lmodern}        % latin modern font
\renewcommand{\familydefault}{\sfdefault}
\usepackage{helvet}         % helvetica as the origin of arial
\usepackage[helvet]{./ETH_res/sfmath} % for the mathematical enviroments
\usepackage{microtype}      % for character protrusion and font expansion (only with pdflatex)

% colors
\usepackage{xcolor}         % color package; load before tocstyle
\usepackage{wallpaper}      % background picture
\setlength{\wpXoffset}{0mm}
\setlength{\wpYoffset}{-10mm}

% page structure
\usepackage{calc}% working with lengths, counters etc.
\usepackage[includeheadfoot,
            top=-0.5mm, bottom=3.5mm,
            left=5.5mm, right=5.5mm,
            headsep=10.5mm, footskip=4.5mm]{geometry}% set page layout parameters
\usepackage{scrpage2}% package for page style with not only uppercase letters in the head
\usepackage{titlesec}% for reducing space between ((sub)sub)sections and text
\usepackage{tocstyle}% for adjusting table of contents
% === toc style ===
\makeatletter
    \newtocstyle[noonewithdot]{nodotnopagenumber}{
    % define tocstyle without dots and page numbers
            \settocfeature{pagenumberbox}{\@gobble}
    }
\makeatother
\usetocstyle{nodotnopagenumber}

% misc
\usepackage{amsthm}     % theorem environments
\usepackage{bm}         % for bold math symbols
\usepackage{enumitem}   % for automatic numbering of new enumerate environments
\usepackage{graphicx}   % for including figures
%\graphicspath{{pictures/}}

\usepackage{tikz}       % sophisticated graphics package
\usepackage{tabularx}   % for special table environment (tabularx-table)
\usepackage{booktabs}   % for table layout
\usepackage{natbib}     % for bibliography with astron-style
\bibliographystyle{astron}
\usepackage{siunitx}    % to use for international units in the real world

\usepackage[
    hypertexnames=false,% for correct links (duplicate-error solution)
	setpagesize=false,  % necessary in order to not change text-/paperformat for the document
	pdfborder={0 0 0},  % removes border around links
	pdfpagemode=FullScreen,% open pdf in full screen mode
    pdfstartview=Fit    % fit page to pdf viewer
]{hyperref}% all links stay black and are thus invisible
%\hypersetup{% Note: If \ETHauthor or \ETHtitle contain non-us-ascii-chars you
%            % should not use them inside \hyperref.
%  pdfauthor=\ETHauthor,
%  pdftitle=\ETHtitle
%}  % docymentsyle and packages
% !TEX TS-program = pdflatex
% !TEX encoding = Isolatin
% !TeX root = ..\ETH_praesentation.tex


%% Version 1.2 21.04.2011
% ==== settings page style and foot ============================================
\pagestyle{scrheadings} % activates pagestyle from scrpage2
\clearscrheadfoot       % clear head and foot of scrheadings and scrplain
                        % setting for page head and foot
\setkomafont{pageheadfoot}{\normalfont\color{black}\sffamily}
% optical vertical centering of page contents
\makeatletter
    \renewcommand*{\@textbottom}{\vskip \z@ \@plus 1fil}
    \newcommand*{\@texttop}{\vskip \z@ \@plus .5fil}
    \addtolength{\parskip}{\z@\@plus .25fil}% stretch parskip a lot
\makeatother
% ==== head ====================================================================
\setheadwidth[0pt]{textwithmarginpar}
\clearscrheadfoot
% head (optional argument is scrplain, mandatory scrheadings)
\lohead{\begin{minipage}[t]{40mm}
              \vspace{0mm}
              \includegraphics[width=0.7\textwidth]{ETH_res/ETH_logo.pdf}
        \end{minipage}
}%

\rohead{\begin{minipage}[t]{35.5mm}
               \vspace*{-1mm}\renewcommand{\baselinestretch}{0.4}
               \bf
               \fontsize{8}{9}\selectfont
               \ETHdeptshort\\\\
               \fontsize{4}{5}\selectfont
               \ETHdeptlong\\
               \ETHinstitut
               \end{minipage}
        }
% ==== foot ====================================================================
\newlength{\footheight}
\setlength{\footheight}{6mm}
\setkomafont{pagefoot}{\normalfont\sffamily\color{white}}
\addtokomafont{pagefoot}{\tiny}% general setting for the foot
\setkomafont{pagenumber}{\color{white}}% setting for page foot
\ifoot{% foot inside (left) to include the figur
	\hspace{-6.0mm}
    \vspace{-8.5mm}
	\begin{tikzpicture}[remember picture,overlay]
		\node [xshift=\paperwidth/2,yshift=\footheight/2]
        {\includegraphics[width=\paperwidth]{ETH_res/ETH_footer.jpg}};
	\end{tikzpicture}%
}%

\cfoot{% foot center to include the text
    \ifnum\value{page} > 1            % first page without footer writings
        \vspace{-6.5mm}
        {\fontsize{5}{6}\selectfont   % very small font
            \begin{minipage}[b]{50mm} % left
                \ETHdate
            \end{minipage}
            \begin{minipage}[b]{57mm} % center
                \ETHauthor
            \end{minipage}
            \begin{minipage}[b]{8mm}  % right
                \hfill \pagemark
            \end{minipage}
        }
    \fi
}%
% ==== end settings page style and foot ========================================

% theorems, proofs
\newtheoremstyle{ETHthmstyle}%
	{0.5em}% space above
	{0.5em}% space below
	{}% body font
	{}% indent amount
	{\sffamily\bfseries}% head font
	{}% punctuation after head
	{\newline}% space after head
	{\thmname{#1}\ \thmnote{(#3)}}% head spec
\theoremstyle{ETHthmstyle}
\newtheorem{theorem}{Theorem}[section]
\newtheorem{remark}[theorem]{Remark}
\newtheorem{algorithm}[theorem]{Algorithm}
\renewcommand*\proofname{Proof}
\makeatletter% correct qed adjustment
    \renewenvironment{proof}[1][\proofname]{\par
    \pushQED{\qed}
    \normalfont\topsep2\p@\@plus2\p@\relax
    \trivlist
    \item[\hskip\labelsep
	      \sffamily\bfseries #1]\mbox{}\hfill\\*\ignorespaces
          }{\popQED\endtrivlist\@endpefalse
    }
\makeatother
% note environment
\newenvironment{note}[1]{
  \labeling{#1}
		\item[#1]\ignorespaces
    }{\endlabeling
    }

% ==== commands ================================================================

% decoration
\usetikzlibrary{decorations.pathmorphing,shadows}


% box
\newcommand*{\ETHbox}[2]{% width, content
	\par\noindent
	\begin{tikzpicture}[ETHnodestyle/.style={rectangle,rounded
            corners,draw=ETHdarkgray!40!white,
            thick,inner sep=2mm,text justified,top color=ETHlightblue!30!white, bottom color=ETHlightblue,above}]%
		    \node[ETHnodestyle,at={(0.5*#1+2mm+0.4pt,0)}]{%
			\begin{minipage}[t]{#1}
				#2
			\end{minipage}%
		    };
%		    \node[ETHnodestyle,at={(0.5*#1+2mm+0.4pt,-0.1)},yscale=-0.33,scope
%                  fading=south,opacity=0.85]{%
%			     \begin{minipage}[t]{#1}
%				        \strut
%			     \end{minipage}%
%		          };
	\end{tikzpicture}%	
	\par\vspace{-1.3em}
}

% labels
\newcommand*{\ETHitem}{\raisebox{1pt}{\tikz\fill[sharp
    corners](0,0)rectangle+(3.4pt,3.4pt);}}

% spacings & make the title section fonts slightly smaller
\titleformat{\section}{\large\bfseries}{\thesection}{1em}{}
\titlespacing{\section}{0mm}{0mm}{0mm}
% make space (left, before, after) between section and text smaller
\titlespacing{\subsection}{0mm}{0mm}{-1mm}
% make space (left, before, after) between subsection and text smaller
\titlespacing{\subsubsection}{0mm}{0mm}{-2mm}
% make space (left, before, after) between subsubsection and text smaller
\setcounter{secnumdepth}{2}% add numbering down to subsection  % settings

% ====  colors  ================================================================
\definecolor{ETHblue}{RGB}{51,89,148}           % #335994
\definecolor{ETHlightblue}{RGB}{114,129,192}    % for Department-Logo
\definecolor{ETHtitleblue}{RGB}{0,49,91}        % for title (#00315b)
\definecolor{ETHbrown}{RGB}{148,116,51}
\definecolor{ETHred}{RGB}{161,82,71}
\definecolor{ETHdarkgray}{rgb}{0.282,0.322,0.361}    % for box (#48525c)
\newcommand*{\ETHblue}[1]{\textcolor{ETHblue}{#1}}
\newcommand*{\ETHlightblue}[1]{\textcolor{ETHlightblue}{#1}}
\newcommand*{\ETHtitleblue}[1]{\textcolor{ETHtitleblue}{#1}}
\newcommand*{\ETHbrown}[1]{\textcolor{ETHbrown}{#1}}
\newcommand*{\ETHred}[1]{\textcolor{ETHred}{#1}}
\newcommand*{\ETHdarkgray}[1]{\textcolor{ETHdarkgray}{#1}}

% ==== document related info ===================================================
\newcommand*{\ETHdeptshort}{\textsc {\color{ETHlightblue}D}\,MATH}
\newcommand*{\ETHdeptlong}{Department of Mathematics}
\newcommand*{\ETHinstitut}{Seminar for Statistics}
\newcommand*{\ETHtitle}{A Non-Parametric Independence Test}
\newcommand*{\ETHsubtitle}{The Hilbert-Schmidt Independence Criterion}
\newcommand*{\ETHauthor}{D MacKinlay, C. Pelloni, A. Toletti / \ETHdeptlong / \ETHinstitut}
\newcommand*{\ETHdate}{Monday, 24th April 2013}

% ==== document language =======================================================
% American English
\usepackage[american]{babel}
%\selectlanguage{american}

% we want these for nice mathematical typesetting
\usepackage{amssymb,amsmath,amsthm,bbm}
\newtheorem*{defn}{Definition}

\AtBeginDocument{\renewcaptionname{american}{\contentsname}{\large Outline}}% toc-name
%% new German
%\selectlanguage{ngerman}
%\AtBeginDocument{\renewcaptionname{ngerman}{\contentsname}{\large \UTF{2039}bersicht}}% toc-name

% ==== document ================================================================
\begin{document}
% ==============================================================================
% titlepage 1
% background picture
\begin{tikzpicture}[overlay]
    \node [shift={(-69.5mm,-133.95mm)}]  at (current page.north east)
    %% ETH-Hauptgebaeude (main building of the ETH)
    {\includegraphics[width=\paperwidth]{ETH_res/ETH_HG.jpg}};
\end{tikzpicture}%
% content title
\begin{flushright}
    %\vspace{-7mm}
    %{\bfseries\Large\ETHtitle\par}
\end{flushright}
	\vspace{-46.5mm}
	{\bfseries\Large{\color{ETHtitleblue}\ETHtitle}}\par
    \vspace{-3.5mm}
    \Large\ETHsubtitle
\clearpage
%%==============================================================================
%% titlepage 2,  please use as an alternative title page
%% background picture
%\begin{tikzpicture}[overlay]
%    \node [shift={(-69.5mm,-133.95mm)}]  at (current page.north east)
%    %% ETH-Polyterrasse (ETH Polyterrace)
%    {\includegraphics[width=\paperwidth]{ETH_res/ETH_PT.jpg}};
%\end{tikzpicture}%
% content title
%\begin{flushright}
%	%\vspace{-7mm}
%	%{\bfseries\Large\ETHtitle\par}
%\end{flushright}
% 	\vspace{-46.5mm}
%	{\bfseries\Large\ETHtitle}\par
%    \vspace{-3.5mm}
%    \Large\ETHsubtitle
%\clearpage
% ==============================================================================
\vspace*{-16.5mm}
\tableofcontents
\clearpage
% ==============================================================================

%#### Please fill in your normal Latex document and add some clearpage-commands.
\section{What are we doing today?}
\begin{enumerate}
	\item We consider the problems with dependency of the PC algorithm on the partial correlation algorithm, which assumes linear, additive noise, then \ldots
	\item\ldots construct an alternative test of independence which can be used instead.
\end{enumerate}
\clearpage
Constructing our alternative will involve
\begin{enumerate}
	\item some functional analysis background,
	\item actually deriving the test, then,
	\item constructing estimator from data
\end{enumerate}
\clearpage

\section{The problem with the partial correlation test}

\subsection{Linearly-additive Gaussian noise models and their limitations}
Last week we saw two methods for inferring graph structures from a sample of variables.

This method presumed linear additive Gaussian noise. However, we are not guaranteed that our data will be anything like that.

\clearpage
Consider:
\begin{enumerate}
\item Categorical data \emph{(If it's raining/if the sprinkler is on/\ldots)}
\item Non-gaussian noise \emph{(e.g. exponential)}
\item Non-additive coupling \emph{(e.g. multiplicative)}
\item Whatever combination of exotic data types, noise and interactions that you would like\dots \emph{(Nucleotide sequences with Bernoulli noise?)}
\end{enumerate}
\clearpage
\section{Examples}
\emph{TODO}
\clearpage
\section{Alternative, non-parametric tests of independence}
What independence tests can relax the strong assumptions of partial correlation?

There are many! e.g. mutual information, rank statistics, copula measures\ldots
\clearpage
\subsection{Hilbert-Schmidt Independence Criterion}
This week's a``kernelisation''-based~\footnote{as made famous by Support Vector Machines} method, has attractive features.
\begin{enumerate}
\item It can handle arbitrary input spaces, not just $\mathbb{R}^d$ (e.g. nucleotide strings)
\item It is easy to conditionalize.
\end{enumerate}
Both those things said, for simplicity, today we will only construct the marginal (unconditional) independence measure on $\mathcal{X}=\mathbb{R}^d$, by way of introduction. (Homework: conditional estimator on general input spaces.)
\clearpage
\section{Functional analysis in 15 minutes}
We consider the prospect of transforming our data in $\mathcal{X}$ into a Hilbert space $\mathcal{H}$, in the hope of finding a convenient representation of the data. This Hilber space, which we will think of as a ``feature space'', will turn out to be a convenient way to represent a wide variety of possible relationships between variables.

\subsection{Hilbert Space}
\begin{defn}
A Hilbert space $\mathcal{H}$ is a complete, non-necessarily finite-dimensional, vector space endowed with an inner product $\langle\cdot,\cdot\rangle: \mathcal{H}\times\mathcal{H} \rightarrow\mathbb{R}$, which fulfils the usual requirements of inner products on vector spaces.
\end{defn}
\clearpage
\begin{defn}
An \emph{inner product}, $\langle\cdot,\cdot\rangle$ on a Hilbert space generalises the definition on $\mathbb{R}^d$ in a natural fashion, in that, for $f,\,g$ and $h \in \mathcal{H}$ and for $\alpha$ and $\beta \in \mathbb{R}$:
\begin{enumerate}
\item $\langle\alpha f + \beta g, h\rangle = \alpha\langle f, h\rangle + \beta\langle g, h\rangle$
\item $\langle f, g\rangle = \langle g, f \rangle$
\item $\langle f, f \rangle \ge 0$ and $\langle f, f\rangle = 0$ iff $f =0$
\end{enumerate}
\end{defn}

Consider an ``input'' metric space $\mathcal{X}$ endowed with $\sigma$-algebra, $\Gamma$,
on a probability space $(\mathcal{X}, \Gamma, \mu)$.
Samples $x_i \in \mathcal{X}$ are drawn independently.
\emph{TODO note Jonas used $\Omega$ here - why?}
\clearpage
We define define a \emph{kernel}, $k$,
\begin{equation}
k : \mathcal{X} \times \mathcal{X} \rightarrow \mathbb{R}
\end{equation}
and a map
\begin{equation}
\phi: \mathcal{X} \rightarrow \mathcal{H}
\end{equation}
such that 
\begin{equation}
k(x, x') = \langle \phi(x), \phi(x') \rangle
\end{equation}
\clearpage

\clearpage
\clearpage
\clearpage
% ===== special page ===========================================================
% ===== page with a background picture =========================================
\ThisCenterWallPaper{1}{ETH_res/ETH_neutral.jpg}
\textcolor{white}{ % change the text on the background picture to white
Discussion, results, conclusion 
}


\ifoot{}% switch off the i-footer for this page used in ETH_res/ETH_settings.tex
\clearpage
%% ===== page after a background picture page ==================================
%\ifoot{% switch on the i-footer for all following pages again
%	\hspace{-6.0mm}
%    \vspace{-8.5mm}
%	\begin{tikzpicture}[remember picture,overlay]
%		\node [xshift=\paperwidth/2,yshift=\footheight/2]
%        {\includegraphics[width=\paperwidth]{ETH_res/ETH_footer.jpg}};
%	\end{tikzpicture}
%}%
%
%% now on normal text again
%More discussion 

\end{document} 